\documentclass[../lecture1-introduction.tex]{subfiles}

\begin{document}

\section{Programming}

% -------------------------------------------------------------------

\begin{frame}[fragile]{What is a Computer Program?}
    A computer program is a collection of instructions that performs a specific
    task when executed by a computer. A computer requires programs to function,
    and typically executes the program's instructions in a central processing unit.

    A part of a computer program that performs a well-defined task is known as
    an algorithm. A collection of computer programs, libraries and related data
    are referred to as software.
\end{frame}

% -------------------------------------------------------------------

\begin{frame}[fragile]{Problem Solving}
    Recipe to writing programs:
    \begin{enumerate}
        \item Understand the problem.
        \item Think of a solution.
        \item Describe the solution in as much detail as possible.
        \begin{verbatim}
            You may use diagrams or plain English to do this.
        \end{verbatim}
        \item Translate your solution into a program.
        \item Run your program and see if it works.
        \begin{itemize}
            \item Yes? Hurray! Victory!
            \item No? Go back to 1
    \end{enumerate}
\end{frame}

% -------------------------------------------------------------------

\begin{frame}[fragile]{Example}
    Think like a computer!

    What steps do you need to take to draw a smiley face?
\end{frame}

% -------------------------------------------------------------------

\end{document}
