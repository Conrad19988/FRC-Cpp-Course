\documentclass[../lecture1-introduction.tex]{subfiles}

\begin{document}

\section{C++}

% -------------------------------------------------------------------

\begin{frame}[fragile]{The C++ Programming Language}
    If you visit \href{http://www.stroustrup.com/C++.html}{stroustrup.com/C++},
    you will come across a plethora of information about the C++ programming
    language, direct from the designer of the language, Bjarne Stroustrup. \newline
    \newline
    Bjarne lists a definition of C++ as: \newline \newline
    "... a general-purpose programming language with a bias towards systems
    programming that:
    \begin{itemize}
        \item Is a better C,
        \item Supports data abstraction,
        \item Supports object-oriented programming, and
        \item Supports generic programming."
    \end{itemize}
\end{frame}

% -------------------------------------------------------------------

\begin{frame}[fragile]{Brief History of C++}
    \begin{itemize} \pause
        \item The language started in 1979 and was originally known as C with Classes. \pause
        \item Essentially it meant that class files (used in object-oriented
        programming), were added to the C language. \pause
        \item In 1983 it was renamed to C++. \pause
        \item C++ exists under the stewardship of a standards committee and
        became an ISO standard in 1998 with a revision in 2011 and a minor
        revision in 2014. \pause
    \end{itemize}
\end{frame}

% -------------------------------------------------------------------

\begin{frame}[fragile]{Example C++ Program (Hello World)}
    \begin{cppcode}[]
#include <iostream>

int main()
{
    cout << "Hello World!";
    return 0;
}
    \end{cppcode}

    \begin{commandshell}
Hello World!
    \end{commandshell}
\end{frame}

% -------------------------------------------------------------------

\end{document}
