\documentclass[../lecture2-variablesandcontrolstructures.tex]{subfiles}

% make sure you use floats instead of doubles. it sounds nuts but if you
% dereference a pointer to a double on the RoboRIO it will crash the kernel

\begin{document}

\section{Data Types}

% -------------------------------------------------------------------

\begin{frame}[fragile]{Identifiers}
    A C++ identifier is a name used to identify a variable, function, class,
    module, or any other user-defined item. \newline \newline
    An identifier starts with a letter A to Z or a to z or an underscore (\_)
    followed by zero or more letters, underscores, and digits (0 to 9). \newline \newline
    C++ does not allow punctuation characters such as \@, \$, and \% within
    identifiers. C++ is a case-sensitive programming language. Thus, Manpower
    and manpower are two different identifiers in C++.
\end{frame}

% -------------------------------------------------------------------

\begin{frame}[fragile]{Reserved Words}
    C++ has a whole table of reserved words (also known as keywords). \newline

    Since they are used by the language, these keywords are not available for
    re-definition or overloading. \newline

    The following list contains some of the reserved words you might come across:
    \begin{table}
        \center
        \begin{tabular}{cccc} %
            \toprule
            and & auto & bool & break \\
            case & catch & char & class \\
            const & continue & default & delete \\
            do & double & dynamic\_cast & else \\
            num & export & extern & false \\
            float & for & if & int \\
            import & long & module & mutable \\
            namespace & new & private & protected \\
            \bottomrule
        \end{tabular}
    \end{table}
    The full list with explanations can be found at \href{http://en.cppreference.com/w/cpp/keyword}{cppreference.com}
\end{frame}

% -------------------------------------------------------------------

\begin{frame}[fragile]{Data Types}
    C++ has several built-in as well as user defined data types to choose from. \newline
    some of the basic C++ data types are:
    \begin{table}
        \center
        \begin{tabular}{c|c|c|c}
            \toprule
            \textbf{Type} & \textbf{Keyword} & \textbf{Width} & \textbf{Range} \\
            \midrule
            Boolean & bool & 1 Byte & true, false \\
            Character & char & 1 Byte & -128 to 127 or 0 to 255 \\
            Integer & int & 4 Bytes & -2,147,483,648 to 2,147,483,647 \\
            Floating point & float & 4 Bytes & $\pm3.4\cdot10^{\pm38}$ (\textasciitilde7 digits) \\
            Double floating point & double & 8 Bytes & $\pm1.7\cdot10^{\pm308}$ (\textasciitilde15 digits) \\
            Valueless & void & & \\
            \bottomrule
        \end{tabular}
    \end{table}
\end{frame}

% -------------------------------------------------------------------

\begin{frame}[fragile]{Choosing Data Types}
t
\end{frame}

% -------------------------------------------------------------------

\end{document}
