\documentclass[../lecture1-introduction.tex]{subfiles}

\begin{document}

\section{Editing, Compiling, and Execution}
% http://www-h.eng.cam.ac.uk/help/languages/C++/c++_tutorial/editing.html
% -------------------------------------------------------------------

\begin{frame}[fragile]{A Simple Program to Add Two Numbers}
    The following is an example of a simple program written in C++.
    % Animate this, first slide to guess what it does, second explains.
    The progam is designed to read two numbers typed by a user at the keyboard;
    compute their sum and display the result on the screen.
    \begin{cppcode}
        // Program to add two integers typed by user at keyboard
        #include <iostream>
        using namespace std;

        int main()
        {
           int a, b, total;

           cout << "Enter integers to be added:" << endl;
           cin >> a >> b;
           total = a + b;
           cout << "The sum is " << total << endl;

           return 0;
        }
    \end{cppcode}
\end{frame}

% -------------------------------------------------------------------

\begin{frame}[fragile]{Program Structure and Syntax}
    C++ uses notation that may appear strange to non-programmers (and me).
    The notation is part of the programming language \textt{syntax}.
    \begin{description}
        \item [Syntax] Formal rules that specify the structure of a legal program.
    \end{description}
\end{frame}

\begin{frame}[fragile]{Program Structure and Syntax}
    the notation and explanations which follow will appear strange if you have
    never written a computer program.

    Don't worry about them or how the program works. This will be explained
    in more detail later.

    The following is an overview.
\end{frame}

\begin{frame}[fragile]{Program Structure and Syntax}
    Every C++ program consists of a header and a main body and has the following
    structure:
    \begin{cppcode}
        // Comment statements which are ignored by computer; just for the reader.
        /* Also a comment */
        #include < header file name >

        int main()
        {
            declaration of variables;
            statements;

            return 0;
        }
    \end{cppcode}
\end{frame}

\begin{frame}[fragile]{Program Structure and Syntax}
    \begin{cppcode}
        // Program to add two integers typed by user at keyboard
        #include <iostream>
        using namespae std;

        int main()
        {
            int num1, num2, total;

            cout << "Enter integers to be added:" << endl;
            cin >> num1 >> num2;
            total = num1 + num2;
            cout << "The sum is " << total << endl;

            return 0;
        }
    \end{cppcode}
    \onslide<1>
    {
        Line 1:
        \begin{itemize}
            \item Lines beginning with // indicate that the rest of the line
            is a \bold{comment}.
            \item Comments are inserted by programmers to help people read
            and understand the program.
            \item Can be placed anywhere in a program.
        \end{itemize}
    }
    \onslide<2>
    {
        Line 2:
        \begin{itemize}
            \item Lines beginning with \# are instructions to the compiler's
            preprocessor.
            \item The \bold{include} instruction says "what follows is a file name,
            find that file and insert its contents right here".
            \item Here the file iostream contains the definitions of
            \bold{cin}, \bold{cout}.
        \end{itemize}
    }
    \onslide<3>
    {
        Line 3:
        \begin{itemize}
            \item Specifies that names used in the program (ie. \bold{cin} and
            \bold{cout}) are defined in the standard libraries.
            \item This is used to avoid problems with other libraries which may
            also use these names.
        \end{itemize}
    }
    \onslide<3>
    {
        Line 5:
        \begin{itemize}
            \item When the program is executed the instructions will be executed
            in the oder they appear in the main body of the program.
            \item 
        \end{itemize}
    }
    \onslide<3>
    {
        \begin{itemize}
            \item
        \end{itemize}
    }
    \onslide<3>
    {
        \begin{itemize}
            \item
        \end{itemize}
    }
\end{frame}

% -------------------------------------------------------------------

\begin{frame}[fragile]{Development Environment \& Development Cycle}



\end{frame}

% -------------------------------------------------------------------

\end{document}
